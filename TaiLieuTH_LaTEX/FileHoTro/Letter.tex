\documentclass{letter} %viết thư

%Hỗ trợ gõ tiếng việt
\usepackage[utf8]{inputenc}
\usepackage[vietnam]{babel}

%chữ ký cuối thư
\signature{Khăn Đỏ} 

%địa chỉ người gởi
\address{227 Nguyễn Văn Cừ \\ Quận 1 \\ Tp Hồ Chí Minh, Việt Nam} 
%ngày tháng sẽ tự động được thêm vô

\begin{document}

%Bắt đầu một lá thư
\begin{letter}{ Khăn Quàng Đỏ\\Bà ngoại của Khăn Đỏ\\ Nhà ở trong rừng\\Kế bên hồ  nước \& Trước nhà có cây ớt} %Danh tính người nhận

%Lời chào đầu thư
\opening{Bà ngoại thân mến:}

%Nội dung thư, chú ý chỗ các khoảng trắng và xuống dòng tùy tiện sẽ ko được hiển thị
		Cháu		mới     nấu 
xong 
món
cà-ri chuối 
	và có nấu cho bà 1 thau. 
Cháu sẽ mang 			cho bà 	vào ngày mai. 

%Muốn bắt đầu 1 đoạn mới thì chừa 1 dòng trắng
Hy vọng bà ăn được.\\[0.7cm] %hoặc sử dụng 2 dấu xuyệt để xuống dòng
Moaz!

\closing{Cháu yêu của bà} %Lời chào cuối thư

%Tái bút
\ps{Tái bút: Nếu ăn không hết thì bà cho chó sói ăn đừng bỏ uổng!}

%Tài liệu gởi kèm theo thư
\encl {Công thức nấu món cà-ri chuối\\ Hình ảnh món cà-ri chuối\\ Địa chỉ bệnh viện gần nhất} 
\end{letter}

%YÊU CẦU 1: Viết tiếp lá thư thứ 2 thì viết tiếp ở đây
%\begin{letter}{ ... }

\end{document}
